%!TEX program = xelatex
\documentclass[a4papers]{ctexart}
\usepackage[english]{babel}
%数学符号
\usepackage{amssymb}
\usepackage{amsmath}
\usepackage{amsthm}
%表格
\usepackage{graphicx,floatrow}
\usepackage{array}
\usepackage{booktabs}
\usepackage{makecell}
%页边距
\usepackage{geometry}
\geometry{left=2cm,right=2cm,top=2cm,bottom=2cm}

%首行缩进两字符 利用\indent \noindent进行控制
\usepackage{indentfirst}
\setlength{\parindent}{2em}

\setromanfont{Songti SC}
%\setromanfont{Heiti SC}

\title{Cryptography -- Homework 1}
\author{冯诗伟161220039}
\date{}
\begin{document}
\maketitle
\section{Problem 1}
\noindent Prove Lemma 2.6 : \\ Encryption scheme Π is perfectly secret if and only if it is perfectly indistinguishable.
\begin{proof}
\[  \mathsf{PrivK}^{\mathsf{eva}}_{\mathcal{A},\Pi}  \]
I. Given that an encryption scheme $\Pi$ is perfectly secret. Let's do the adversial indistinguishable experiment ${\rm PrivK}^{eva}_{\mathcal{A},\Pi}$.
Now the adversary $\mathcal{A}$ knows the ciphertext $c$ and he has to output a bit, 0 or 1. We have
\[{\rm Pr}[M=m_0] ={\rm Pr}[M=m_1] = \dfrac{1}{2}\]
Due to the perfectly secrecy,
\[{\rm Pr}[M=m_0 | C=c] ={\rm Pr}[M=m_0] = \dfrac{1}{2}\]
\[{\rm Pr}[M=m_1 | C=c] ={\rm Pr}[M=m_1] = \dfrac{1}{2}\]
Thus, it is obvious that\[{\rm Pr}[\mathsf{PrivK}^{\mathsf{eva}}_{\mathcal{A},\Pi}=1] = \dfrac{1}{2} \]
which means $\Pi$ is perfectly indistinguishable.\\
II. Given that an encryption scheme $\Pi$ is perfectly indistinguishable. It means that even if I know the ciphertext of one of the two
plaintext, the probability of success is exactly $\frac{1}{2}$ which is the same as that of just guessing.
\[{\rm Pr}[M=m_0 | C=c] ={\rm Pr}[M=m_1 | C=c] = \dfrac{1}{2}\]
\[\therefore{\rm Pr}[M=m_i | C=c] ={\rm Pr}[M=m_i] = \dfrac{1}{2},\, i=0,1.\]
Therefore, $\Pi$ is perfectly secret.
\end{proof}

\section{Problem 2}
\noindent Prove or refute: Every encryption scheme for which the size of the key space equals the size of the message space, 
and for which the key is chosen uniformly from the key space, is perfectly secret.
\begin{proof}
    Refute. These restrictions only focus on the size and distribution and do not pay attention to matters about distinguishability.
Imagine an encryption scheme where $\mathcal{K}=\mathcal{M}= \{ 0,1\} ^{ \ell }$ and the key is uniformly chosen from $\mathcal{K}$ 
while $\mathsf{Enc}_{K}(m)=m$.
This encryption scheme is not secret at all.
\end{proof}

\section{Problem 3}
\noindent (a) Solution:
\begin{alignat*}{2}
    Pr[\mathsf{PrivK}^{\mathsf{eva}}_{\mathcal{A},\Pi}=1] 
    &= \dfrac{1}{2}Pr[\mathsf{PrivK}^{\mathsf{eva}}_{\mathcal{A},\Pi}=1 | b=0] + \dfrac{1}{2}Pr[\mathsf{PrivK}^{\mathsf{eva}}_{\mathcal{A},\Pi}=1 | b=1]\\
    &= \dfrac{1}{2}Pr[\mathcal{A}\, {\rm output}\, 0 | b=0 ] + \dfrac{1}{2}Pr[\mathcal{A} \,{\rm output}\, 1 | b=1 ]\\
    &= \dfrac{1}{2}Pr[\mathcal{A}\, {\rm output}\, 0 | b=0 ] +  \dfrac{1}{2}\Big(1-Pr[\mathcal{A}\, {\rm output} \, 0|b=1) \Big)\\
    &= \dfrac{1}{2}\Big(\dfrac{1}{3}Pr[\mathcal{A}\, {\rm output}\, 0 | b=0,\,t=1 ] 
                      + \dfrac{1}{3}Pr[\mathcal{A}\, {\rm output}\, 0 | b=0,\,t=2 ]
                      + \dfrac{1}{3}Pr[\mathcal{A}\, {\rm output}\, 0 | b=0,\,t=3 ]\Big) \\
    &+ \dfrac{1}{2}\Big(1-\big( \dfrac{1}{3}Pr[\mathcal{A}\, {\rm output}\, 0 | b=1,\,t=1 ] 
                      + \dfrac{1}{3}Pr[\mathcal{A}\, {\rm output}\, 0 | b=1,\,t=2 ] 
                      + \dfrac{1}{3}Pr[\mathcal{A}\, {\rm output}\, 0 | b=1,\,t=3 ]\big) \Big)\\
    &= \dfrac{1}{2}\Big(\dfrac{1}{3}\times 1+\dfrac{1}{3}\times\dfrac{26}{26^2}+\dfrac{1}{3}\times\dfrac{26\times 1\times 26}{26^3} \Big)
      + \dfrac{1}{2}\Big(1-\big(\dfrac{1}{3}\times 0+\dfrac{1}{3}\times\dfrac{26}{26^2}+\dfrac{1}{3}\times\dfrac{26\times 1\times 26}{26^3} \big)\Big)\\
    &= \dfrac{2}{3}
\end{alignat*}

\noindent (b) Solution:
Define $\mathcal{A'}$ as follows: $\mathcal{A'}$ output $m_0=aa$ and $m_1=ab$.
When give a ciphertext $c=c_1 c_2$, it outputs 0 if $c_1=c_2$, and outputs 1 if $c_1\ne _2$.
\begin{alignat*}{2}
    Pr[\mathsf{PrivK}^{\mathsf{eva}}_{\mathcal{A'},\Pi}=1] 
    &= \dfrac{1}{2}Pr[\mathsf{PrivK}^{\mathsf{eva}}_{\mathcal{A'},\Pi}=1 | b=0] + \dfrac{1}{2}Pr[\mathsf{PrivK}^{\mathsf{eva}}_{\mathcal{A'},\Pi}=1 | b=1]\\
    &= \dfrac{1}{2}Pr[\mathcal{A'}\, {\rm output}\, 0 | b=0 ] + \dfrac{1}{2}Pr[\mathcal{A'} \,{\rm output}\, 1 | b=1 ]\\
    &= \dfrac{1}{2}Pr[\mathcal{A'}\, {\rm output}\, 0 | b=0 ] +  \dfrac{1}{2}\Big(1-Pr[\mathcal{A'}\, {\rm output} \, 0|b=1) \Big)\\
    &= \dfrac{1}{2}\Big( \dfrac{1}{2}Pr[\mathcal{A'}\, {\rm output}\, 0 | b=0,\, t=1 ]+\dfrac{1}{2}Pr[\mathcal{A'}\, {\rm output}\, 0 | b=0,\, t=2 ]\Big)\\
    &+ \dfrac{1}{2}\Big( 1 - \big( \dfrac{1}{2} Pr[\mathcal{A'}\, {\rm output} \, 0|b=1\, t=1) +  \dfrac{1}{2} Pr[\mathcal{A'}\, {\rm output} \, 0|b=1\, t=2)\big)\Big)\\
    &= \dfrac{1}{2}\Big(\dfrac{1}{2}\times 1+\dfrac{1}{2}\times\dfrac{1}{26} \Big)
      + \dfrac{1}{2}\Big(1-\big(\dfrac{1}{3}\times 0+\dfrac{1}{2}\times\dfrac{1}{26} \big)\Big)\\
    &= \dfrac{3}{4} > \dfrac{2}{3}
\end{alignat*}
\end{document}


















