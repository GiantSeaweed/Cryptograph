%!TEX program = xelatex
\documentclass[a4papers]{ctexart}
%数学符号
\usepackage{amssymb}
\usepackage{amsmath}
%表格
\usepackage{graphicx,floatrow}
\usepackage{array}
\usepackage{booktabs}
\usepackage{makecell}
%页边距
\usepackage{geometry}
\geometry{left=2cm,right=2cm,top=2cm,bottom=2cm}

%首行缩进两字符 利用\indent \noindent进行控制
\usepackage{indentfirst}
\setlength{\parindent}{2em}

\setromanfont{Songti SC}
%\setromanfont{Heiti SC}

\title{Cryptography -- Homework 1}
\author{冯诗伟161220039}
\date{}
\begin{document}
\maketitle
\section{8.2}

\section{8.4}
解:
\indent 设这批钢索的断裂强度为$X$千克/平方厘米,$X \sim N(\mu,40^2)$,记原假设$H_0$和备择假设$H_1$分别为
\[
    H_0: \mu =\overline x - 20 \quad H_1 :\mu < \overline x - 20
    \]
检验统计量为$\dfrac{\overline X -\mu}{\sigma / \sqrt{n}}$,当$H_0$成立时,
$U=\dfrac{\overline X -(\overline x - 20)}{\sigma / \sqrt{n}}\sim N(0,1)$。则检验的拒绝域为
\[
    W = \left\{ U=\dfrac{\overline X -(\overline x - 20)}{\sigma / \sqrt{n}} \le -u_{\alpha} \right\}
    \]
其中,$\overline X=\overline x,\sigma = 40,n=9,-u_{\alpha}=-u_{0.01}=-2.33,$
检验统计量的观察值$u=1.5\notin W,$所以接受原假设,认为该批钢索的断裂强度有所提高。

\section{8.6}

\end{document}