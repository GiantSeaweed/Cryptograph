%!TEX program = xelatex
\documentclass[a4papers]{ctexart}
%数学符号
\usepackage{amssymb}
\usepackage{amsmath}
\usepackage{amsthm}
%表格
\usepackage{graphicx,floatrow}
\usepackage{array}
\usepackage{booktabs}
\usepackage{makecell}
%页边距
\usepackage{geometry}
\geometry{left=2cm,right=2cm,top=2cm,bottom=2cm}

%首行缩进两字符 利用\indent \noindent进行控制
\usepackage{indentfirst}
\setlength{\parindent}{2em}

\setromanfont{Songti SC}
% \setromanfont{Heiti SC}
\newcommand{\mc}[1]{\mathcal{#1}}
\newcommand{\ms}[1]{\mathsf{#1}}

\title{Cryptography--Homework 3}
\author{冯诗伟161220039}
\date{}
\begin{document}
\maketitle
\section*{1}
\subsection*{a}
No. Construct a new message $m'=m_2||m_1||m_3||\cdots||m_\ell$ by swapping the first two blocks of $m$ and use $m'$ to query the oracle.
We can get $t'=\mc{O}(m')=\ms{Mac_k}(m')=F_k(m_2)\oplus F_k(m_1)\oplus F_k(m_3)\oplus \cdots \oplus F_k(m_\ell)=\ms{Mac_k}(m)$.
So we successfully forge a valid tag $t'$ such that $\ms{Vrfy_k}(m,t')=1$ and $(m,t')\notin \mc{Q}$.

\subsection*{b}
No. Given $m=m1||m2$, we can easily find two messages $m1', m2'(m1'\ne m1,\ m2'\ne m2)$. Then we can construct two new messages $m_A$ and $m_B$, where
$m_A=m1'||m2$, $m_B=m1||m2'$. 

Query the oracle with $m_A$, we can get $\mc{O}(m_A)=F_k(m_1')||F_k(F_k(m_2))$.
Query the oracle with $m_B$, we can get $\mc{O}(m_b)=F_k(m_1)||F_k(F_k(m_2'))$.
By concatenating the former half of $\mc{O}(m_b)$ with the latter half of $\mc{O}(m_A)$, we can forge a valid tag $t=F_k(m_1)||F_k(F_k(m_2))=\ms{Mac}_k(m)$, where $\ms{Vrfy_k}(m,t)=1$ and $(m,t)\notin \mc{Q}$.

\subsection*{c}
No. Given $m=m_1||m_2||m_3||\cdots||m_\ell$, we can construct the following three messages:
\[m_A=m_1||m_2||m_2||\cdots||m_\ell\]
\[m_B=m_2||m_2||m_2||\cdots||m_\ell\]
\[m_C=m_2||m_2||m_3||\cdots||m_\ell\]

Then we query the oracle with these three new messages and produce a tag $t$ by 
XOR the three responses.
\begin{alignat*}{2}
    t &= \mc{O}(m_A)\oplus \mc{O}(m_b)\oplus \mc{O}(m_c)\\
    &= F_k(<1>|m_1)\oplus F_k(<2>|m_2)\oplus F_k(<3>|m_2)\oplus \oplus^{\ell}_{i=4} F_k(<i>|m_i)\\
    &\oplus  F_k(<1>|m_2)\oplus F_k(<2>|m_2)\oplus F_k(<3>|m_2)\oplus \oplus^{\ell}_{i=4} F_k(<i>|m_i)\\
    &\oplus  F_k(<1>|m_2)\oplus F_k(<2>|m_2)\oplus F_k(<3>|m_3)\oplus \oplus^{\ell}_{i=4} F_k(<i>|m_i)\\
    &= F_k(<1>|m_1)\oplus F_k(<2>|m_2)\oplus F_k(<3>|m_3)\oplus \oplus^{\ell}_{i=4} F_k(<i>|m_i)\\
    &= F_k(<1>|m_1)\oplus F_k(<2>|m_2)\oplus F_k(<3>|m_3)\oplus \cdots \oplus F_k(<\ell>|m_\ell)\\
    &= \ms{Mac}_k(m)
\end{alignat*}

So we successfully forge a valid tag $t$ where $\ms{Vrfy_k}(m,t)=1$ and $(m,t)\notin \mc{Q}$.

\subsection*{d}
No. Given $m=m_1||m_2||m_3||\cdots||m_\ell$, we can first construct the following messages:
\[ m_A=m_1||m_2||m_2||\cdots||m_\ell \]

Then we query the oracle with $m_A$ and get the response:
\begin{alignat*}{2}
\mc{O}(m_A)&=(r,m_r)\\
    &=(r,\, F_k(r)\oplus F_k(<1>|m_1)\oplus F_k(<2>|m_2)\oplus F_k(<3>|m_2)\oplus \oplus^{\ell}_{i=4} F_k(<i>|m_i) )
\end{alignat*}

We can parse $r$ as $r=<x>|r'$, where the former half is the $\frac{n}{2}$-bit encoding of the integer $x$.

Construct the second message:
\[ m_B=m_2||m_2||m_2||\cdots||r'||\cdots||m_\ell \]
where $r'$ is the $x$-th block and $\ell=2^{\frac{n}{2}}$. We can query the oracle with $m_A$ mutiple times until $x\ge4$.

Then we query the oracle with $m_B$ and get the response:
\begin{alignat*}{2}
  \mc{O}(m_B)&=(p,m_p)\\
  &=(p,\, F_k(p)\oplus F_k(<1>|m_2)\oplus F_k(<2>|m_2)\oplus F_k(<3>|m_2)\oplus F_k(<x>|r')\oplus \oplus^{2^{\frac{n}{2}}}_{i=5} F_k(<i>|m_i) )
\end{alignat*}

Similarly, We can parse $p$ as $p=<y>|p'$, where the former half is the $\frac{n}{2}$-bit encoding of the integer $y$.

Construct the third message:
\[ m_C=m_2||m_2||m_3||\cdots||p'||\cdots||m_\ell \]
where $p'$ is the $y$-th block and $\ell=2^{\frac{n}{2}}$. We can query the oracle with $m_B$ mutiple times until $y\ge4$.

Then we query the oracle with $m_C$ and get the response:
\begin{alignat*}{2}
  \mc{O}(m_C)&=(q,m_q)\\
  &=(q,\, F_k(q)\oplus F_k(<1>|m_2)\oplus F_k(<2>|m_2)\oplus F_k(<3>|m_3)\oplus F_k(<y>|p')\oplus \oplus^{2^{\frac{n}{2}}}_{i=5} F_k(<i>|m_i) )
\end{alignat*}

Notice that $F_k(<x>|r')=F_k(r),\,F_k(<y>|p')=F_k(p)$, we can produce the tag $t=(q,M)$ and 
\begin{alignat*}{2}
    M &= m_r \oplus m_q \oplus m_q\\
    &= F_k(r)\oplus F_k(<1>|m_1)\oplus F_k(<2>|m_2)\oplus F_k(<3>|m_2)\oplus \oplus^{\ell}_{i=4} F_k(<i>|m_i) \\
    &\oplus F_k(p)\oplus F_k(<1>|m_2)\oplus F_k(<2>|m_2)\oplus F_k(<3>|m_2)\oplus F_k(<x>|r')\oplus \oplus^{2^{\frac{n}{2}}}_{i=5} F_k(<i>|m_i) \\
    &\oplus F_k(q)\oplus F_k(<1>|m_2)\oplus F_k(<2>|m_2)\oplus F_k(<3>|m_3)\oplus F_k(<y>|p')\oplus \oplus^{2^{\frac{n}{2}}}_{i=5} F_k(<i>|m_i)\\
    &=F_k(q)\oplus F_k(<1>|m_1)\oplus F_k(<2>|m_2)\oplus F_k(<3>|m_3)\oplus \oplus^{\ell}_{i=4} F_k(<i>|m_i) \\
    &\oplus F_k(r)\oplus F_k(<x>|r') \oplus F_k(p)\oplus  F_k(<y>|p') \\
    &=F_k(q)\oplus F_k(<1>|m_1)\oplus F_k(<2>|m_2)\oplus F_k(<3>|m_3)\oplus \oplus^{\ell}_{i=4} F_k(<i>|m_i) 
\end{alignat*}

The discussion above is based on the condition that $x$ and $y$ are greater than 4.
\begin{alignat*}{2}
    \Pr[x\ge 4 \wedge y\ge 4]&=\Pr[x\ge 4]\Pr[y\ge4]\\
    &= (1-2^{\frac{n}{2}-2})^2
\end{alignat*}

So we can construct an adversary $\mc{A}$ conducting the steps mentioned above so that
\[\Pr[\ms{Mac}-\ms{sforge}_{\mc{A},\Pi}] = 1 = (1-2^{\frac{n}{2}-2})^2\]
which is non-negligible.

With a non-negligible probability we successfully forge a valid tag $t=(q,M)$ where $\ms{Vrfy_k}(m,t)=1$ and $(m,t)\notin \mc{Q}$, 
so this $\ms{MAC}$ is not strongly secure.
\section*{2}
\subsection*{a}
This is not collision-resistant.
Construct two messages:
\[x=x_1||x_2||\cdots||x_{B-1}||0^n\]
\[x'=x_1||x_2||\cdots||x_{B-1}||0^{n-1}\]
$x^i \in \{0,1\}^{n},\,|x|=Bn,|x'|=Bn-1$. Because the first step is to pad the messages
with zero so its length is a multiple of $n$, $x$ will remain unchanged and $x'$ will be padded with one zero 
and become the same as $x$. It is obvious that $H^s(x)$ will equal $H^s(x')$ , 
which serves as an attack.

\subsection*{b}
This is collision-resistant.
We show that for any $s$, a collision in modified $H^s$ yields a collision in $h^s$, 
thereby proving that the modified version is collision-resistant.

Let $x$ and $x'$ be two different strings of length $L$ and $L'$ respectively, such that
$H^s(x)=H^s(x')$. Let $x_1,\cdots,x_B$ be the $B$ blocks of the padded $x$, and let $x_1',\cdots,x_B'$ 
be the $B'$ blocks of the padded $x'$. Recall that $x_{B+1}=L$ and $x'_{B+1}=L'$. 
There are two cases to consider:

1. Case1: $L\ne L'$. In this case, the last step of the computation of $H^s(x)$ is $z_{B+1}=z_{B}||L$, and
the last step of the computation of $H^s(x')$ is $z'_{B'+1}=z'_{B'}||L'$. Since $H^s(x)=H^s(x')$ it follows that 
$z_{B}||L = z'_{B'}||L'$. However, $L \ne L'$ and so $z_{B}||L$ and $z'_{B'}||L'$ are two different strings that collide under $h^s$.

2. Case2: $L=L'$. This means that $B=B'$. Let $z_0,\cdots,z_{B+1}$ be the values defined during
the computation of $H^s(x)$, let $I_i \overset{def}{=} z_{i-1}||x_i$ denote the $i$-th input to $h^s$, $1\le i\le B+1$,
 and set $I_{B+2}\overset{def}{=}z_{B+1}=z_B||L$. Define $I_1',\cdots,I_{B+2}'$ analogously with respect to $x'$.
 Let $N$ be the largest index for which $I_N\ne I_{N}'$. Since $|x|=|x'|$ but $x\ne x'$, there is an $i$ with $x_i\ne x_i'$
 and so such that an $N$ certainly exists. Because
 \[ I_{B+2}=z_{B+1}=H^s(x)=H^s(x')=z_{B+1}'=I_{B+2}' \]
 we have $N\le B+1$. By maximality of $N$, we have $I_{N+1}=I_{N+1}'$ and in particular $z_{N}=z_{N}'$. 
 This means that $I_N,\, I_N'$ are a collision in $h^s$.

 In conclusion, the modified Merkle-Damgård Tranform is collision-resistant.


\subsection*{c}
This is collision-resistant.
We show that for any $s$, a collision in modified $H^s$ yields a collision in $h^s$, 
thereby proving that the modified version is collision-resistant.

Let $x$ and $x'$ be two different strings of length $L$ and $L'$ respectively, such that
$H^s(x)=H^s(x')$. Let $x_1,\cdots,x_B$ be the $B$ blocks of the padded $x$, and let $x_1',\cdots,x_B'$ 
be the $B'$ blocks of the padded $x'$. Recall that $x_{B+1}=L$ and $x'_{B+1}=L'$. 
There are two cases to consider:

1. Case1: $L\ne L'$. In this case, the last step of the computation of $H^s(x)$ is $z_{B+1}=z_{B}||L$, and
the last step of the computation of $H^s(x')$ is $z'_{B'+1}=z'_{B'}||L'$. Since $H^s(x)=H^s(x')$ it follows that 
$z_{B}||L = z'_{B'}||L'$. However, $L \ne L'$ and so $z_{B}||L$ and $z'_{B'}||L'$ are two different strings that collide under $h^s$.

2. Case2: $L=L'$. This means that $B=B'$. Let $z_0,\cdots,z_{B+1}$ be the values defined during
the computation of $H^s(x)$, let $I_i \overset{def}{=} z_{i-1}||x_i$ denote the $i$-th input to $h^s$, $2\le i\le B+1$,
 and set $I_1 = z_1 = x_1,\, I_{B+2}\overset{def}{=}z_{B+1}=z_B||L$. Define $I_1',\cdots,I_{B+2}'$ analogously with respect to $x'$.
 Let $N$ be the largest index for which $I_N\ne I_{N}'$. Since $|x|=|x'|$ but $x\ne x'$, there is an $i$ with $x_i\ne x_i'$
 and so such that an $N$ certainly exists. Because
 \[ I_{B+2}=z_{B+1}=H^s(x)=H^s(x')=z_{B+1}'=I_{B+2}' \]
 we have $N\le B+1$. By maximality of $N$, we have $I_{N+1}=I_{N+1}'$ and in particular $z_{N}=z_{N}'$. 
 This means that $I_N,\, I_N'$ are a collision in $h^s$.

 In conclusion, the modified Merkle-Damgård Tranform is collision-resistant.


\subsection*{d}
This is not collision-resistant.
Consider the meassage $x = x_1||x_2,\, |x_1|=|x_2|=n,$,$z_0=2n$ .
Let $x_2$ be an $n$-bit encoding of the integer $n$.

Suppose $h^s(x1||z_0)=z_1$, and construct the message $x' = z_1, z_0'=|x'|=n$.

\[ H^s(x')=h^s(x'||z_0')\overset{*}{=}h^s(x'||x_2)=h^s(z_1||x_2)=H^s(x)\]

*: both $z_0'$ and $x_2$ are $n$-bit encoding of the integer $n$.


 $h^s(x1||z_0)=z_1$, and suppose that $z_1$ is an $n$-bit encoding of the integer $\ell_1$.

 If $\ell_1\ne 0 $, we can construct a second message $x',\,|x'|=\ell_1$. It is obvious that $H^s(x_1||x')=H^s(x')$.
 
 If $\ell_1 = 0$, consider $h^s(x_2||z_1)=z_2$ and suppose that $z_2$ is an $n$-bit encoding of the integer $\ell_2$.
 There must be $\ell_2\ne \ell_1 = 0$, because $z_1 \ne z_0$($z_1$ represent $0$ and $z_0$ represent 2$n$) and $h^s$ is 
 conllision resistant.
 we can construct a message $x'',\,|x''|=\ell_2$. It is obvious that $H^s(x_1||x_2||x'')=H^s(x'')$.

\section*{Additional 3.26}
Define $\Pi=(Gen,H),\,\Pi_1=(Gen_1,H_1),\,\Pi_2=(Gen_2,H_2)$.
Suppose $\Pi$ is not collision-resistant. Then there exists an PPT adversary $\mc{A}$ 
that $\mc{A}$ can found a collision in $H$ with a non-negligible probability $\epsilon(n)$:
\[ \Pr[\ms{Hash-coll}_{\mc{A},\Pi}=1]=\epsilon(n)\]

Now we can construct $\mc{A}_1$ with $\mc{A}$:

$\mc{A}_1$ is given $s_1,s_2$.

1. Run $\mc{A}(s_1,s_2)$ and obtain $x, \ x'$.

2. Output $x,\,x'$.

$\mc{A}_1$ runs in polynomial time since $\mc{A}$ does. 
Whenever $\mc{A}$ found a collision in $H$, $\mc{A}_1$ found a collision in $H_1$. 
Also, $\mc{A}_2$ can find a collision in $H_2$. So we have 
\[ \Pr[\ms{Hash-coll}_{\mc{A}_1,\Pi_1}=1] > \Pr[\ms{Hash-coll}_{\mc{A},\Pi}=1] = \epsilon(n) \]
\[ \Pr[\ms{Hash-coll}_{\mc{A}_2,\Pi_2}=1] > \Pr[\ms{Hash-coll}_{\mc{A},\Pi}=1] = \epsilon(n) \]

So we have that both $\mc{A}_1$ and $\mc{A}_2$ are not collision-resistant.

In conclusion, if at least one of $H_1$ and $H_2$ are collision-resistant, $H$ is collision-resistant.

\end{document}