%!TEX program = xelatex
\documentclass[a4papers]{ctexart}
%数学符号
\usepackage{amssymb}
\usepackage{amsmath}
\usepackage{amsthm}
%表格
\usepackage{graphicx,floatrow}
\usepackage{array}
\usepackage{booktabs}
\usepackage{makecell}
%页边距
\usepackage{geometry}
\geometry{left=2cm,right=2cm,top=2cm,bottom=2cm}

%首行缩进两字符 利用\indent \noindent进行控制
\usepackage{indentfirst}
\setlength{\parindent}{2em}

\setromanfont{Songti SC}
% \setromanfont{Heiti SC}

\title{Cryptography--Homework 2}
\author{冯诗伟161220039}
\date{}
\begin{document}
\maketitle
\section*{3.14}
\noindent If $F$ is a length-preserving pseudorandom function, we have
 \[|Pr\big[ D^{F_k(·)}(1^n)=1\big] - Pr\big[ D^{f(·)}(1^n)=1\big]| \le negl(n)\]
 It means that no efficient adversary can distinguish $F_k$ from $f$.\\
 Go back to $G(s) =F_s(1)||F_s(2)||···||F_s(\ell)$. \\
 Consider a uniform string $r \in \{ 0,1 \}^{\ell·n}$. For a fixed $r$, it will be the output of $f$ with a probability of $2^{-\ell·n}$.
 Meanwhile, it will be the output of $G(s)$ with a probability of $(2^{-n})^\ell = 2^{-\ell·n}$ since $F_k(x)$ and $F_k(y)$ are independent if $x\ne y$.\\
 Therefore, for any efficient distinguisher, there will be 
 \[|Pr[D(G(s))=1]-Pr[D(r)=1]| \le negl(n)\]
 Notice that $G(s)$ is deterministic. So $G(s)$ is a pseudorandom generator with an expansion factor of $\ell·n$.

\section*{3.19}
\subsection*{a}
The encryption scheme is not EAV-secure. Because $G$ is deterministic and publicly known, the adversary can get the plaintext
just by compute $G(c_1)\oplus c_2 = G(r) \oplus G(r) \oplus m = m$ when obtaining the ciphertext $c=<c_1,c_2>$.

The encryption scheme is not CPA-secure for the reason mentioned above.

\subsection*{b}
The encryption scheme is EAV-secure. Even if the input of $F$ is fixed, the key $k$ is uniform. So it is equivalent to OTP.

The encryption scheme is not CPA-secure. After the adversary produces $m_0$ and $m_1$, he gets $m_b \oplus F_k(0^n)$. Then the adversary 
can sent $m_0$ to the oracle and get $m_0 \oplus F_k(0^n)$ from the oracle. Since $k$ is the same, the adversary now gets the value of $F_k(0^n)$. 
Thus he can directly compute the $m_b$ and succeeds with the probability of 1.

\subsection*{c}
The encryption scheme is both EAV-secure and CPA-secure.

Because $m_0$ and $m_1$ are independent and $F_k(r)$ and $F_k(r+1)$ are independent, this is actually the same as Construction3.30. 
Therefore, it is CPA-secure and we get that it is EAV-secure for free. (Actually, this scheme is CTR-mode.)

\section*{3.29}
\noindent Construct the CPA-secure encryption scheme $\Pi^*=(Enc,Dec)$:\\
$Enc$: on input a message $m \in \{0,1\}^n$, choose a uniform $r\in \{0,1\}^n$ and output the ciphertext
\[ c:= <Enc_1(r),Enc_2(r\oplus m)>\]
$Dec$: on input a ciphertext $c = <c_1,c_2>$, output the plaintext message
\[m:=Dec_1(c_1) \oplus Dec_2(c_2)\]
Next I will show that $\Pi^*$ is CPA-secure as long as at least one of $\Pi_1$ or $\Pi_2$ is CPA-secure.\\
1. If both $Enc_1$ and $Enc_2$ are CPA-secure, the adversary can get nothing about $r$ or $r\oplus m$. So he cannot 
get any information about the plaintext.\\
2. If $Enc_1$ is CPA-secure while $Enc_2$ is not CPA-secure, the adversary can distinguish $r\oplus m_0$ from $r\oplus m_1$ with a probability significantly better than $\frac{1}{2}$.
But $r$ is uniform, the adversary cannot tell $m_0$ from $m_1$.\\
3. If $Enc_2$ is CPA-secure while $Enc_1$ is not CPA-secure, the adversary can get some information about $r$.
That is no help because the adversary can get nothing about $r\oplus m$.\\
In conclusion, $\Pi^*$ is CPA-secure as long as at least one of $\Pi_1$ or $\Pi_2$ is CPA-secure.\\

\section*{Additional 3.26}
\subsection*{a}
\noindent If $F$ is pseudorandom, we have
 \[|Pr\big[ D^{F_k(·)}(1^n)=1\big] - Pr\big[ D^{f(·)}(1^n)=1\big]| \le negl(n)\]
 Since $f(·)$ is originally uniform, $f^\$(·)$ is just the same as $f(·)$.
 \[Pr\big[ D^{f(·)}(1^n)=1\big] = Pr\big[ D^{f^\$(·)}(1^n)=1\big]\]
Since $F$ is pseudorandom, $F$ with uniform $k$ actually produces random strings. Thus $F_k$ with uniform input produces random strings as well.
\[Pr\big[ D^{F_k(·)}(1^n)=1\big] = Pr\big[ D^{F^\$_k(·)}(1^n)=1\big]\]
Therefore, we have 
 \[|Pr\big[ D^{F^\$_k(·)}(1^n)=1\big] - Pr\big[ D^{f^\$(·)}(1^n)=1\big]| \le negl(n)\]
 which means $F$ is weakly pseudorandom.

\subsection*{b}
\noindent Notice that whether the distinguisher $D$ for $F'_k$ can succeed is independent from the parity of $x$.
\begin{alignat*}{2}
    Pr\big[ D^{F^\$_k(·)}(1^n)=1\big] 
    &= Pr\big[ D^{F'^\$_k(·)}(1^n)=1 \cap x \,is\, even \big] + Pr\big[ D^{F'^\$_k(·)}(1^n)=1 \cap x \,is\, odd \big]\\
    &= \dfrac{1}{2} Pr\big[ D^{F'^\$_k(·)}(1^n)=1] + \dfrac{1}{2}Pr\big[ D^{F'^\$_k(·)}(1^n)=1]\\
    &= Pr\big[ D^{F'^\$_k(·)}(1^n)=1]
\end{alignat*}
Because $F'$ is pseudorandom, $F'$ is weakly pseudorandom.
\[ |Pr\big[ D^{F'^\$_k(·)}(1^n)=1] - Pr\big[ D^{f(·)}(1^n)=1\big]| \le negl(n) \]
Notice that \[Pr\big[ D^{f(·)}(1^n)=1\big] = Pr\big[ D^{f^\$(·)}(1^n)=1\big],\, 
Pr\big[ D^{F'^\$_k(·)}(1^n)=1] = Pr\big[ D^{F^\$_k(·)}(1^n)=1\big] \]
\[\therefore  |Pr\big[ D^{F^\$_k(·)}(1^n)=1] - Pr\big[ D^{f^\$(·)}(1^n)=1\big]| \le negl(n) \]
which means $F$ is weakly pseudorandom.\\
Next I will prove that $F$ is not pseudorandom.\\
Construct the distinguisher $D^*$: $D^*$ query the oracle with $m_0$ and $m_1$ where $m_0$ is even, $m_1$ is odd
 and $m_0 = m_1 + 1$. $D^*$ obtains $y_0 = \mathcal{O}(m_0)$ and $y_1 = \mathcal{O}(m_1)$ and outputs 1 if and only if
 $y_0 = y_1$. \\
 If $\mathcal{O} = F_k$, $D^*$ outputs 1 with probability of 1.
 If $\mathcal{O} = f$, $D^*$ outputs 1 with probability of $2^{-n}$. 
 \[ \therefore |Pr\big[ D^{F_k(·)}(1^n)=1] - Pr\big[ D^{f(·)}(1^n)=1\big]| = 1-2^{-n} \]
 So $F_k$ is not pseudorandom.

\subsection*{c}
\noindent CTR-mode encryption using a weak pseudorandom function is not necessarily CPA-secure.\\
Consider the weak pseudorandom function $F_k$ in (b).\\
Construct the adversary $\mathcal{A}$.
$\mathcal{A}$ produces $m_0$ and $m_1$ where $m_0$ and $m_1$ are as long as three blocks. \\
$m0 = m_{00}||m_{01}||m_{02},$ where $m_{00} = m_{01} = m_{02}$.\\
$m1 = m_{10}||m_{11}||m_{12},$ where any two of $m_{10}, m_{11}, m_{12}$ are different.\\
$\mathcal{A}$ obtains the ciphertext $c=c_{b1} || c_{b2} || c_{b3}$ and can get the parity of ctr.\\
If ctr is odd, $\mathcal{A}$ output 0 if and only if $c_{b1}= c_{b2}$.\\
If ctr is even, $\mathcal{A}$ output 0 if and only if $c_{b1}= c_{b2}$.\\
Notice that when ctr is odd $c_{01} = m_{01} \oplus F_k(ctr+3), c_{02} = m_{02} \oplus F_k(ctr+3),c_{11}\ne c_{12}$,
and when ctr is even $c_{00} = m_{00}\oplus F_k(ctr+2), c_{01} = m_{01} \oplus F_k(ctr+2),c_{00}\ne c_{01}$.\\
$\mathcal{A}$ succeeds with a probability of 1.\\
In this case, CTR-mode encryption using a weak pseudorandom function is not CPA-secure.\\


\noindent CTR-mode encryption using a weak pseudorandom function is not necessarily EAV-secure.\\
Consider the weak pseudorandom function $F_k$ in (b).\\
Construct the adversary $\mathcal{A}$.
$\mathcal{A}$ produces $m_0$ and $m_1$ where $m_0$ and $m_1$ are as long as three blocks. \\
$m0 = m_{00}||m_{01}||m_{02},$ where $m_{00} = m_{01} = m_{02}$.\\
$m1 = m_{10}||m_{11}||m_{12},$ where any two of $m_{10}, m_{11}, m_{12}$ are different.\\
$\mathcal{A}$ obtains the ciphertext $c=c_{b1} || c_{b2} || c_{b3}$ and can get the parity of ctr.\\
If ctr is odd, $\mathcal{A}$ outputs 0 if and only if $c_{01}=c_{02}$.\\
If ctr is even, $\mathcal{A}$ outputs 0 if and only if $c_{00}=c_{01}$.\\
Notice that when ctr is odd $c_{01} = c_{02} ,c_{11}\ne c_{12}$,
and when ctr is even $c_{00} = c_{01} ,c_{00}\ne c_{01}$.\\
$\mathcal{A}$ succeeds with a probability of 1.\\
In this case, CTR-mode encryption using a weak pseudorandom function is not EAV-secure.\\


\subsection*{d}
If $F_k$ is weakly pseudorandom, we have 
\[ |Pr\big[ D^{F^\$_k(·)}(1^n)=1] - Pr\big[ D^{f^\$(·)}(1^n)=1\big]| \le negl(n)\]
Let $\widetilde{\Pi}=(\widetilde{Gen},\widetilde{Enc},\widetilde{Dec})$ be an encryption scheme that is exactly the same
as $\Pi = (Gen,Enc,Dec)$ from Construction3.30, except that a truly random function is used in the place of $F_k$. \\
Let $\Pi^\$=(Gen^\$,Enc^\$,Dec^\$)$ be an encryption scheme that is exactly the same
as $\Pi = (Gen,Enc,Dec)$ from Construction3.30, except that a weakly pseudorandom function is used in the place of $F_k$. \\ \\
Construct the distinguisher $\mathcal{D}$:\\
$\mathcal{D}$ is given input $1^n$ and access to an oracle $\mathcal{O}:\{0,1\}^n \rightarrow \{0,1\}^n$.\\
1. Run $\mathcal{A}(1^n)$. Whenever $\mathcal{A}$ queries its encryption oracle on a message $m \in\{0,1\}^n $,
answer this query in the following way:\\
\indent (a) Choose uniform $r \in\{0,1\}^n $.\\
\indent (b) Return the ciphertext $<r,\mathcal{O}(r)\oplus m> $ to $\mathcal{A}$.\\
2. When $\mathcal{A}$ outputs messages $m_0,m_1 \in\{0,1\}^n $, choose a uniform bit $b \in\{0,1\}$ and then:\\
\indent (a) Choose uniform $r \in\{0,1\}^n $.\\
\indent (b) Return the ciphertext $<r,\mathcal{O}(r)\oplus m> $ to $\mathcal{A}$.\\
3. Continue answering the encryption-oracle queries of $\mathcal{A}$ as before until $\mathcal{A}$ outputs a bit $b'$.
Output 1 if $b=b'$ and 0 otherwise.\\ \\
If $\mathcal{D}$'s oracle is a random function,
\[ Pr_{f\leftarrow Func_n}\big[ D^{f(·)}(1^n)=1\big] = Pr\big[ Privk^{cpa}_{\mathcal{A},\widetilde{\Pi}}(n)=1\big]\]
If $\mathcal{D}$'s oracle is a weakly pseudorandom function,
\[ Pr_{k\leftarrow \{0,1\}^n}\big[ D^{F^\$_k(·)}(1^n)=1\big] = Pr\big[ Privk^{cpa}_{\mathcal{A},\Pi^\$}(n)=1\big]\]
\[\because |Pr\big[ D^{F^\$_k(·)}(1^n)=1] - Pr\big[ D^{f^\$(·)}(1^n)=1\big]| \le negl(n),\,\,
Pr\big[ D^{f^\$(·)}(1^n)=1\big] = Pr\big[ D^{f(·)}(1^n)=1\big] \]
\[\therefore | Pr\big[ Privk^{cpa}_{\mathcal{A},\Pi^\$}(n)=1\big]-Pr\big[ Privk^{cpa}_{\mathcal{A},\widetilde{\Pi}}(n)=1\big]| \le negl(n) \]
Recall the inequation(3.11) in the textbook:
\[ Pr\big[ Privk^{cpa}_{\mathcal{A},\widetilde{\Pi}}(n)=1\big]\le \dfrac{1}{2} + \dfrac{q(n)}{2^n}\]
We now have
\[Pr\big[ Privk^{cpa}_{\mathcal{A},\Pi^\$}(n)=1\big] \le \dfrac{1}{2} + \dfrac{q(n)}{2^n} + negl(n)\]
Since $q(n)$ is polynomial, $\dfrac{q(n)}{2^n}$ is negligible, which means Construction3.30 is CPA-secure if $F$ is a weak pseudorandom fucntion.
\end{document}