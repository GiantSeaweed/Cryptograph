%!TEX program = xelatex
\documentclass[a4papers]{ctexart}
%数学符号
\usepackage{amssymb}
\usepackage{amsmath}
\usepackage{amsthm}
%表格
\usepackage{graphicx,floatrow}
\usepackage{array}
\usepackage{booktabs}
\usepackage{makecell}
%页边距
\usepackage{geometry}
\geometry{left=2cm,right=2cm,top=2cm,bottom=2cm}

%首行缩进两字符 利用\indent \noindent进行控制
\usepackage{indentfirst}
\setlength{\parindent}{2em}

\setromanfont{Songti SC}
% \setromanfont{Heiti SC}
\newcommand{\mc}[1]{\mathcal{#1}}
\newcommand{\ms}[1]{\mathsf{#1}}
\newcommand{\mr}[1]{\ \mathrm{#1}\ }

\title{Cryptography--Homework 4}
\author{冯诗伟161220039}
\date{}
\begin{document}
\maketitle
\section*{1}
Assume that the adversary $\mathcal{A}$ can break 1$\%$ of $\mathbb{Z}_N^*$ (a specific subset)
with probability of 1 and break the other 99$\%$ with probability of 0.
 
We can construct $\mathcal{A'}$ as follows:

1. Given $y=x^e \mr{mod} N$.

2. Uniformly choose $r \in \mathbb{Z}_N^*$ and $r^{-1}$ such that $r\cdot r^{-1} = 1 \mr{mod} N$.

3. Feed $y\cdot r^e$ to $\mathcal{A}$ and get $z = \mathcal{A}(y\cdot r^e\, \mathrm{mod}\, N)$.

4. Compute $y' = z\cdot r^{-1} \mr{mod} N$.

5. Repeat step 1 to 4 for XXX times. Denote the $y'$ in the $i$-th round as $y'_i$.
If any two outputs($y'_i$ and $y'_j$) in these XXX rounds are the same, output $y'_i$. 
Otherwise output 0.

Let me explain it in detail.

Step1 and Step2 is trival. In Step3, $z=y^{1/e}$

\section*{2}
\section*{Additional}
\end{document}