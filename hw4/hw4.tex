%!TEX program = xelatex
\documentclass[a4papers]{ctexart}
%数学符号
\usepackage{amssymb}
\usepackage{amsmath}
\usepackage{amsthm}
%表格
\usepackage{graphicx,floatrow}
\usepackage{array}
\usepackage{booktabs}
\usepackage{makecell}
%页边距
\usepackage{geometry}
\geometry{left=2cm,right=2cm,top=2cm,bottom=2cm}

%首行缩进两字符 利用\indent \noindent进行控制
\usepackage{indentfirst}
\setlength{\parindent}{2em}

\setromanfont{Songti SC}
% \setromanfont{Heiti SC}
\newcommand{\mc}[1]{\mathcal{#1}}
\newcommand{\ms}[1]{\mathsf{#1}}
\newcommand{\mr}[1]{\ \mathrm{#1}\ }
\newcommand{\mb}[1]{\mathbb{#1}}
\title{Cryptography--Homework 4}
\author{冯诗伟161220039}
\date{}
\begin{document}
\maketitle
\section*{1}
Assume that the adversary $\mathcal{A}$ can break 1$\%$ of $\mathbb{Z}_N^*$ (a specific subset)
with probability of 1 and break the other 99$\%$ with probability of 0.
 
We can construct $\mathcal{A'}$ as follows:

1. Given $y=x^e \mr{mod} N$.

2. Uniformly choose $r \in \mathbb{Z}_N^*$ and $r^{-1}$ such that $r\cdot r^{-1} = 1 \mr{mod} N$.

3. Feed $y\cdot r^e$ to $\mathcal{A}$ and get $z = \mathcal{A}(y\cdot r^e\, \mathrm{mod}\, N)$.

4. Compute $x' = z\cdot r^{-1} \mr{mod} N$. Check whether $(x')^e = y \mr{mod} N$ holds.

5. Repeat step 1 to 4 for 459 times.
 Denote the $x',\, y $ in the $i$-th round as $x'_i, y_i$.
If $(x'_i)^e = y_i \mr{mod} N$ for some $i$, output $x'_i$ and stop. 
If this did not occur after 459 rounds, output $x'_1$.

Let me explain it in detail.

Step1 and Step2 are trival.

In Step3, the equation, $z=(y\cdot r^e)^{1/e}= y^{1/e}\cdot r \mr{mod} N$, holds with probability of $0.01$ since $\mc{A}$ succeeds with probability of $0.01$.

In Step4, if $\mc{A}$ called by $\mc{A'}$ succeeds in Step3, we can get $x' = y^{1/e}\cdot r\cdot r^{-1} = y^{1/e} \mr{mod} N$, which means $(x')^e = y \mr{mod} N$.

In Step5,
%  first we compute the probability that $\mc{A'}$ succeeds in less than $k$ rounds.
% $y'_k$ equals exactly one $y'_i$ for some $i$ < $k$. Denote this event as $E_k$.
intuition is that the more times $\mc{A'}$ tries, the more likely $\mc{A'}$ will succeeds. 
So we want to know the least $k$ that satisfy the following inequation:
\[ 1- (1-0.01)^k \ge 0.99\]
which means after trying $k$ times $\mc{A'}$ will succeeds with a probability not less than $0.99$.
% \[ \sum_{i=2}^k \Pr[E_i] \ge 0.99 \]
% \[ \therefore \sum_{i=2}^k 0.01 \times C_{i-1}^1(0.99)^{i-2} \times 0.01 \ge 0.99\]
So we have:\[ k \ge 459\]
% \[ k \ge 662\]
As to the running time $t'$, Step1 and Step4 is $\mc{O}(1)$. Step2 is $\mc{O}(log\ N)$ and Step3 is $\mc{O}(t)$.
So running time $t'$ of $\mc{A'}$ is polynomial in $t$ and $\lVert N \rVert$.

\section*{2}
First, because $g$ is a generator($g^n = 1\, \mr{mod}\, n $), $(g^r)^n = (g^n)^r = 1^r = 1\, \mr{mod}\, n$ .

Second, we can show that for all $0\le i \le n-1,\, i\ne r$, there exists a $q(0\le q \le n-1)$ such that $(g^r)^q = g^i \mr{mod}\, n$.

Let $(g^r)^q = g^i \mr{mod}\, n$. So $rq = i\, \mr{mod}\, n$. Because $gcd(r,n)=1$, there exists a $r^{-1}$
such that $r\cdot r^{-1} = 1\, \mr{mod}\, n$. So we have
\[ q = i\cdot r^{-1} \, \mr{mod}\, n\]

So far, we have proved that $g^r$ is also a generator of $\mathbb{G}$.

\section*{Additional}
\noindent Construct the following one-way function family 
$\Pi = (\mathsf{Gen},\mathsf{Samp},f)$:

$\mathsf{Gen}$: Given $1^n$, outputs parameters $I=(\mb{G},\, q,\, g)$ where the order $\lVert q \rVert = n$. $g$ is the generator of $\mb{G}$.
$\mc{D}_I = \mc{R}_I = \mb{G}.$

$\mathsf{Samp}$: On input $I$, outputs a uniformly distributed $x\in \mc{D}_I$. 

$f$ : On input $I$ and $x\in \mc{D}_I$, outputs $y=f_I(x) = g^x$.

\noindent We are going to show that $\Pi$ is easy to compute and hard to invert.

1. Easy to compute. Given $g$ and $x$, the complexity of compute $g^x \mr{mod} n$ is $\mc{O}(log\ x)$ .

2. Hard to invert. Design the following experiment 
$\mathsf{Invert}_{\mc{A},\Pi}(n)$:

(1) $\mathsf{Gen}$ is run to obtain $I$, and then $\mathsf{Samp}(I)$ is run to obtain a uniform $x \in \mc{D}$.
Finally $y:=f_I(x) = g^x$ is computed.

(2) $\mc{A}$ is given $I$ and $y = g^x$ as input, and outputs $x'$.

(3) The output if the experiment is 1 if $f_I(x')=x$.

\noindent We can see that the view of $\mc{A}$ in $\mathsf{Invert}_{\mc{A},\Pi}(n)$ is identical to that of 
$\mathsf{DLog}_{\mc{A},\mathsf{Gen}}(n)$. So $\mc{A}$ succeeds in $\mathsf{Invert}_{\mc{A},\Pi}(n)$
if and only if $\mc{A}$ succeeds in $\mathsf{DLog}_{\mc{A},\mathsf{Gen}}(n)$. We have:
\[\Pr[\mathsf{Invert}_{\mc{A},\Pi}(n)=1] 
= \Pr[\mathsf{DLog}_{\mc{A},\mathsf{Gen}}(n)=1] \le negl(n)\]
So we have shown that $\Pi$ is hard to invert.






\end{document}